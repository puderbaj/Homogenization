 \documentclass[10pt, a4paper]{article}
 \usepackage[german,ngerman]{babel}
 \usepackage[latin1]{inputenc}
 \usepackage[T1]{fontenc}
 \usepackage{amsmath}
 \usepackage{amssymb}
\usepackage{pict2e}
\usepackage{tikz}


 
\begin{document}
%TODO: - Motivation
%      - Description of two-scale homogenization/periodicity
%      - Construction of geometry with two scale level set functions
%      - sketch of such an geometry and the process
%      - refine process (Is it commutative? How to save time?)
%      - different types of level set functions (square, circle, ...)
%      - How two handle cut level set functions
%      - Hierarchy of grids 
%      - Memory aspects
%      - Interface conditions

\section{Motivation}

\section{Two-scale periodicity}

We want to create a geometry with two levels of periodicity. 

\begin{figure}
\begin{tikzpicture}
\draw (-1,-1) -- (-1,1) -- (1,1) -- (1,-1) -- (-1,-1);
\draw (-0.5,-0.5) -- (-0.5,0.5) -- (0.5,0.5) -- (0.5,-0.5) -- (-0.5,-0.5);
% \draw (0,0) circle (0.5);

\end{tikzpicture}
\label{first_scale_periodicity_cell_pic}
\caption{First scale periodicity cell}
\end{figure}

\begin{figure}
\begin{tikzpicture}
\draw (-1,-1) -- (-1,1) -- (1,1) -- (1,-1) -- (-1,-1);
% \draw (-0.5,-0.5) -- (-0.5,0.5) -- (0.5,0.5) -- (0.5,-0.5) -- (-0.5,-0.5);
\draw (0,0) circle (0.5);

\end{tikzpicture}
\label{second_scale_periodicity_cell_pic}
\caption{Second scale periodicity cell}
\end{figure}

\begin{figure}
\begin{tikzpicture}
\draw (-4,-4) -- (-4,4) -- (4,4) -- (4,-4) -- (-4,-4);
\foreach \x in {-4,...,3}
{
    \foreach \y in {-4,...,3}
    {
        \draw (\x + 0.25 , \y +0.25) -- ( \x + 0.75 ,\y +0.25) -- (\x + 0.75 , \y + 0.75) -- (\x +0.25,  \y +0.75) -- (\x + 0.25 ,  \y +0.25);
    }
}

\end{tikzpicture}
\end{figure}
Let $\Omega = [0,1]^d$ be a domain with space dimension $d$. 
To create the first level of periodicity we scale and shift the first scale periodicity cell

% \begin{figure}
% 	\begin{tikzpicture}
%   %microscopic structure
%   \draw (1/64,1/64) -- (1/64,133/64) -- (133/64,133/64) -- (133/64,1/64) -- (1/64,1/64);
% 
%   \foreach \x in {1,...,16}
%   {
%     \foreach \y in {1,...,16}
%     {
% 		\fill[gray] (\x/8,\y/8) circle (0.4mm);
% 	}
%   }
% %   \node at (2.5,2.5){\small{microscopic problem}};
% 	\end{tikzpicture}
	
% 	\begin{tikzpicture}
%       \draw (0,0) -- (0,1) -- (1,1) -- (1,0) -- (0,0);
%       \fill[gray] (0.5,0.5) circle (1mm);
% 	\end{tikzpicture}
% 
% 	\begin{tikzpicture}
%       	\fill[gray] (0,0) -- (0,1) -- (1,1) -- (1,0) -- (0,0);
% 	\draw (0,0) -- (0,1) -- (1,1) -- (1,0) -- (0,0);
% 	\end{tikzpicture}
% 	\caption{Microscopic problem }
% \end{figure}
 
 
 
 \end{document}
