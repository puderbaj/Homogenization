\documentclass[10pt]{beamer}	% Is better for some Windows user.
%
\usepackage{pict2e}
\usepackage{tikz}
\usepackage[ngerman]{babel} % Pakete laden
\usepackage[utf8]{inputenc}
\usepackage[T1]{fontenc}
\usepackage{graphicx, xcolor}
% \usepackage{BeamerColor}
\usepackage{tikz, amsfonts, amsmath}
\usetheme{Goettingen}
% \usecolortheme[named=darkgray]{structure}
\def\UDM{ % untere Dreiecks-Matrix
\mbox{
\setlength{\unitlength}{9pt}
\begin{picture}(1,1)
\put(0,1){\line(1,0){1}}
\put(1,0){\line(0,1){1}}
\put(0,1){\line(1,-1){1}}
\end{picture}
}}
\title{Homogenization}
\author{Janna Puderbach}
% \input{docinfo}				% Edit here author informations.
% \input{header}
% \input{titlepage}
% \input{headingsandfooter}
\usepackage{algorithm2e}
\begin{document}
    \addtocounter{framenumber}{-1}	% Exclude page from pagecounter.
	\begin{frame}[plain]
		\titlepage
	\end{frame}
	%
	%\setcounter{framenumber}{0}
	\frame{
		\frametitle{Content}
		\tableofcontents
	}



	\section{Introduction}

\begin{frame}{Introduction} %Motivation
Given a composite material %Bild
\begin{figure}
\begin{tikzpicture}
%microscopic structure
\draw (0.25,0.25) -- (0.25,2.25) -- (2.25,2.25) -- (2.25,0.25) -- (0.25,0.25);

  \foreach \x in {1,...,4}
  {
    \foreach \y in {1,...,4}
    {
		\fill[gray] (\x/2,\y/2) circle (1mm);
	}
  }
		
	\draw[dashed] (1.75,1.75) -- (1.75,1.25) -- (1.25,1.25) -- (1.25,1.75) -- (1.75,1.75);
	
	\draw[|-|] (0.25,0) -- (2.25,0);
	\node at (1.25,-0.25){$L$};
	
	\draw[|-|] (2.5,1.75) -- (2.5,1.25);
	\node at (2.75,1.5){$l$};
	
	\node at (2.75,0.25){$\Omega$};
	
% 	\draw[out=30, in=30] (1.75,1.75) to (4,25,2.25) ;
	
% periodicity cell
\draw (4.25,0.25) -- (4.25,2.25) -- (6.25,2.25) -- (6.25,0.25) -- (4.25,0.25);
\fill[gray] (5.25,1.25) circle (5mm);
\node at (4.75,1.75) {$\sigma_2$};
\node at (5.25,1.25) {$\sigma_1$};
\node at (6.75,0.25) {$\Pi$};
\draw[|-|] (4.25,0) -- (6.25,0);
\node  at (5.25,-0.25){$l$};
	
\end{tikzpicture}
\end{figure}
\begin{itemize}
\item $\Omega$ - domain with scale $L$
\item $\Pi$ - periodic cell with scale $l$
\item $\varepsilon = \frac{l}{L} \ll 1$
\end{itemize}
% \textbf{microscopic scale}
% \begin{itemize}
% \item fine scale of inhomogeneous part
% \item too expensive to solve with FEM
% \end{itemize}
% \textbf{macroscopic scale}
% \begin{itemize}
% \item Consider the material as homogeneous
% \item easy to solve
% \end{itemize}
\end{frame}

% %Example
\begin{frame}
Example (Conductivity Problem)
\begin{figure}
\begin{tikzpicture}

% periodicity cell
\draw (4.25,0.25) -- (4.25,2.25) -- (6.25,2.25) -- (6.25,0.25) -- (4.25,0.25);
\fill[gray] (5.25,1.25) circle (5mm);
\node at (4.75,1.75) {$\sigma_2$};
\node at (5.25,1.25) {$\sigma_1$};
\node at (6.75,0.25) {$\Pi$};
\draw[|-|] (4.25,0) -- (6.25,0);
\node  at (5.25,-0.25){$l$};
	
\end{tikzpicture}
\end{figure}
\begin{align*}
- \operatorname{div} \only<1>{(\sigma(x) \nabla u) &= g(x), \quad & x \in \Omega} \only<2>{(\sigma(x/\varepsilon) \nabla u_{\varepsilon}) &= g(x), \quad & x \in \Omega} \\
\only<1> {u &= 0 , \quad & x \in \partial \Omega} \only<2>{u_\varepsilon &= 0 , \quad & x \in \partial \Omega } \\ \\
\sigma(x) &= \begin{cases} \sigma_1 , x\in \Omega_1\\ \sigma_2, x\in \Omega_2 \end{cases}
\end{align*}

% \begin{tikzpicture}
% \draw  ;
% \end{tikzpicture}
\end{frame}
%

% %formal asymptotic expansion
\begin{frame}{Formal Asymptotic Expansion}
Ansatz:
\begin{align*}
\only<1>{u_{\varepsilon} (x) = u_0(x,\frac{x}{\varepsilon}) + \varepsilon u_1(x,\frac{x}{\varepsilon}) + \varepsilon^2 u_2(x,\frac{x}{\varepsilon}) + ...}
\only<2>{u(x,y) = u_0(x,y) + \varepsilon u_1(x,y) + \varepsilon^2 u_2(x,y) + ...}
\end{align*}
\only<2>{with $y = \frac{x}{\varepsilon}$} \\
\only<2>{Consider $x$ (slow) and $y$ (fast) as independent variables}
\end{frame}

\begin{frame}
Two-scale problem
\begin{align*}
\varepsilon = \frac{l}{L} \ll 1 \\
\end{align*}
$$ \varepsilon \to 0$$
\begin{align*}
-\operatorname{div} (\sigma(x/\varepsilon) \nabla u_{\varepsilon}) &= g(x), \quad & x \in \Omega \\
u_{\varepsilon} &= 0, \quad & \partial \Omega
\end{align*}
\end{frame}
\section{Homogenization problem}
%Homogenized problem
\begin{frame}
\begin{align}
-\operatorname{div} (\hat \sigma \nabla u_0) &= g(x), \quad &x \in \Omega \\
u_0 &=0, \quad &x \in \partial \Omega
\end{align}
\end{frame}
\begin{frame}
\begin{align*}
-\operatorname{div} (\sigma(\frac{x}{\varepsilon}) \nabla u_{\varepsilon}) &= g(x) , \quad & x \in \Omega\\
u_{\varepsilon} &= 0, \quad &x \in \partial \Omega
\end{align*}
\begin{align*}
u_{\varepsilon} &\rightharpoonup u_0 &\text{ weakly in } &H^1_0(\Omega)\\
\sigma_{\varepsilon}\nabla u_{\varepsilon} &\rightharpoonup \hat \sigma \nabla u_0 &\text{ weakly in } &L^2(\Omega)
\end{align*}
\end{frame}

\begin{frame}{Effective Conductivity Tensor}
\begin{align*}
\hat \sigma_{i,j} = \frac{1}{|\Pi|} \int_{\Pi} (\sigma(y)(\nabla \chi_i + \boldsymbol e_i)) \cdot (\nabla \chi_j + \boldsymbol e_j) \operatorname d y
\end{align*}
\end{frame}

\begin{frame}{Cell problem}
\begin{align}
- \operatorname{div} \left[ \sigma(y)[\nabla \chi_i + \boldsymbol e_i ]\right] = 0 , \quad y \in \Pi
\end{align}
\begin{figure}
\begin{tikzpicture}
\draw  (0,0) -- (4,0) -- (4,4) -- (0,4) -- (0,0);
\draw[white] (1,3) circle (0.2cm) node[black]{$\sigma_1$};
\draw[white] (1,1) circle (0.2cm) node[black]{$\Pi$};
\fill[gray] (2,2) circle (1cm) node[white]{$\sigma_2$};
\end{tikzpicture}
\caption{Unit cell}
\end{figure}
\end{frame}
%

%
\section{Two-Scale Convergence}
%Two-scale convergence
\begin{frame}{Two-Scale Convergence}
% \textbf{Strong Convergence}

\textbf{Weak Convergence}


\textbf{Two-Scale convergence}
\begin{align*}
\lim_{\varepsilon \to 0} \int_
{\Omega} u_\varepsilon (x) \Psi(x,\frac{x}{\varepsilon}) \operatorname d x = \int_{\Omega} \int_\Pi u_0(x,y) \Psi(x,y)\operatorname d y \operatorname d x
\end{align*}
\end{frame}


% \begin{frame}
% \begin{itemize}
% \item microscopic problem
% \begin{itemize}
% \item very fine grid necessary to resolve the structure
% \end{itemize}
% \item Homogenized problem
% \begin{itemize}
% \item need to solve extra problem (cell problem)
% \end{itemize}
% \end{frame}
%\begin{frame}{Quellen}

%\end{frame}

% \begin{frame}
% \begin{center}
%     \begin{huge}
%     Vielen Dank für Ihre Aufmerksamkeit!\\[1,5cm]
%     Fragen?
%     \end{huge}
% \end{center}
%
% \end{frame}

\end{document}
