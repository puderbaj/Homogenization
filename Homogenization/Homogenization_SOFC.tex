 \documentclass[10pt, a4paper]{article}
 \usepackage[german,ngerman]{babel}
 \usepackage[latin1]{inputenc}
 \usepackage[T1]{fontenc}
 \usepackage{amsmath}
 \usepackage{amssymb}
\usepackage{pict2e}
\usepackage{tikz}


 \begin{document}
%  \section{Model}
 
 \begin{align}
 \partial_t u &= \nabla \cdot (D_V\nabla u)   & x\in S \label{diffusion}\\
 D_V\nabla u \cdot \nu &= k(u-\bar{u}) & x \in \Gamma \label{boundary condition}
 \end{align}
 
 We assume $D_V$ to be constant in $Y_S$.
 
%  \begin{align*}
% %  D(x) = 
% %  \begin{cases}
% %  D_V \\
% %  0
% %  \end{cases}
%  D(x) = \chi_\varepsilon(x) D_V
%  \end{align*}

 Ansatz:
\begin{align}
u_\varepsilon (x,y) = u_0(x,y) + \varepsilon u_1(x,y) + \varepsilon^2 u_2(x,y) + \mathcal{O} (\varepsilon^3)
\end{align}
 
 
 Chain rule for the gradient:
 \begin{align}
 \nabla = (\nabla_x + \frac{1}{\varepsilon} \nabla_y)
 \end{align}
 
 The left hand side of \eqref{diffusion} becomes:
 
 \begin{align}
 \partial_t u_\varepsilon =
 \partial_t \left( u_0(x,y) + \varepsilon u_1(x,y) + \varepsilon^2 u_2(x,y)\right)
 \end{align}
 
 The right hand side of \eqref{diffusion} becomes:
 
 \begin{align*}
 \nabla &\cdot (D_V \nabla u_\varepsilon) = \nabla \cdot (D_V\nabla(u_0 + \varepsilon u_1 + \varepsilon^2 u_2)) \\
        =& (\nabla_x + \frac{1}{\varepsilon} \nabla_y) \cdot (D_V (\nabla_x + \frac{1}{\varepsilon} \nabla_y)(u_0 + \varepsilon u_1 + \varepsilon^2 u_2))\\
        =& D_V ( \varepsilon^{-2}( \nabla_y \cdot \nabla_y u_0) \\
        &+ \varepsilon^{-1}(\nabla_x \cdot \nabla_y u_0 + \nabla_y \cdot \nabla_x u_0 + \nabla_y \cdot \nabla_y u_1) \\
        &+ \varepsilon^0(\nabla_x \cdot \nabla_x u_0 + \nabla_x \cdot \nabla_y u_1 + \nabla_y \cdot \nabla_x u_1 + \nabla_y \cdot \nabla_y u_2 )\\ 
        &+ \varepsilon^1( \nabla_x \cdot \nabla_x u_1 + \nabla_x \cdot \nabla_y u_2 + \nabla_y \cdot \nabla_x u_2)\\
        &+ \varepsilon^2(\nabla_x \cdot \nabla_x u_2))
 \end{align*}
 Comparing the orders of $\varepsilon$ we get for $\varepsilon^{-2}$:
 \begin{align}
 \nabla_y \cdot \nabla_y u_0(x,y) = 0
 \end{align}
 Because of the periodicity in $y$ it is $u_0(x,y) = u_0(x)$.
 
 Therefore we get from the $\varepsilon^{-1}$:
 \begin{align}
 \nabla_x \cdot \nabla_y u_0(x) = \nabla_y \cdot \nabla_x u_0(x) = 0 
 \end{align}
 
 It follows:
 
 \begin{align}
 \nabla_y \cdot \nabla_y u_1(x,y) = 0
 \end{align}
 
 We write $u_1$ as linear combination of $u_0$
 \begin{align}
 \label{lin_combi_u_1}
 u_1(x,y) = \sum_{j=1}^n w_j(y) \partial_{x_j}u_0(x)
 \end{align}
 The $y$-gradient can be written as
 \begin{align}
 \label{gradienty u1}
 \nabla_y u_1(x,y) = & \nabla_y \sum_{j=1}^n w_j(y) \partial_{x_j}u_0(x)\\
                   = & \sum_{j=1}^n \partial_{x_j} u_0(x) \nabla_y w_j(y) \\
                   = & \sum_{j=1}^n \partial_{x_j} u_0(x) \sum_{i=1} e_i \partial_{y_i} w_j(y)
 \end{align}
 Or component-wise
 \begin{align*}
 \partial_{y_i} u_1(x,y) = \sum_{j=1}^n \partial_{x_j} u_0(x) \partial_{y_i}w_j(y)
 \end{align*}
It is obvious that $\nabla_x u_0(x)$ can be represented as:
  \begin{align}
 \nabla_x u_0(x) = \sum_{i=1}^n e_i \partial_{x_i}u_0(x)
 \end{align}
%  \textit{Kann man das wirklich machen, wenn $D_V = const$? }
 
 
 From the $\varepsilon^0$ term we get:
 
 \begin{align*}
 \partial_t u_0 = D_V & (\nabla_x \cdot \nabla_x u_0 + \nabla_x \cdot \nabla_y u_1 + \nabla_y \cdot \nabla_x u_1 + \nabla_y \cdot \nabla_y u_2)
 \end{align*}
 

 
 We integrate over the cell:
 
 \begin{align*}
 \int_{Y_S} \partial_t u_0 \operatorname{d} y = 
       D_V (& \int_{Y_S} \nabla_x \cdot (\nabla_x u_0 +    \nabla_y u_1) \operatorname{d} y 
       + \int_{Y_S} \nabla_y \cdot (\nabla_x u_1 + \nabla_y u_2) \operatorname{d} y )
 \end{align*}
 
  
 The first integral then becomes:
 
 \begin{align*}
 D_V &\int_{Y_S} \nabla_x \cdot (\nabla_x u_0(x) + \nabla_yu_1(x,y)) \operatorname{d}y \\
 &=  D_V \int_{Y_S} \nabla_x \cdot (\sum_{i=1}^n e_i \partial_{x_i} u_0(x) + \sum_{j=1}^n 
  \partial_{x_j} u_0(x) \sum_{i=1}^n e_i\partial_{y_i}w_j(y)) \operatorname{d} y\\
 &= D_V \int_{Y_S} \sum_{i=1}^n \partial_{x_i}^2 u_0(x) + \sum_{j=1}^n\sum_{i=1}^n 
 \partial_{x_i}\partial_{x_j}u_0(x) \partial_{y_i}w_j(y)\operatorname{d}y\\
 &= D_V \int_{Y_S} \sum_{i=1}^n\sum_{j=1}^n \delta_{ij}\partial_{x_i}\partial_{x_j}u_0(x)
 + \sum_{i=1}^n\sum_{j=1}^n \partial_{x_i}\partial_{x_j} u_0(x) \partial_{y_i}w_j(y)\operatorname{d}y\\
 &= D_V \int_{Y_S} \sum_{i=1}^n\sum_{j=1}^n\partial_{x_i}\partial_{x_j} u_0(x)
 (\delta_{ij} + \partial_{y_i} w_j(y))\operatorname{d}y\\
 &= \sum_{i=1}^n\sum_{j=1}^n\partial_{x_i}\partial_{x_j} u_0(x)
 D_V \int_{Y_S}(\delta_{ij} + \partial_{y_i} w_j(y))\operatorname{d}y
 \end{align*}
 
 We introduce now:
 
\begin{align}
\hat D_{ij} =D_V \int_{Y_S}(\delta_{ij} + \partial_{y_i} w_j(y))\operatorname{d}y
\end{align}
 
 Then we get:
 \begin{align}
 D_V &\int_{Y_S} \nabla_x \cdot (\nabla_x u_0(x) + \nabla_yu_1(x,y)) \operatorname{d}y 
 = \sum_{i=1}^n\sum_{j=1}^n \partial_{x_i}\partial_{x_j}u_0(x) \hat D_{ij}
 \end{align}
 Consider the second integral and integrate by parts:
 
 \begin{align}
 \label{divergence theorem}
 D_V \int_{Y_S} \nabla_y \cdot (\nabla_x u_1 + \nabla_y u_2) \operatorname{d} y = D_V \int_\Gamma (\nabla_x u_1 + \nabla_y u_2) \cdot \nu \operatorname{d} \Gamma(y)
 \end{align}
 
 Putting these two integrals together we get:
 \begin{align}
 \label{integral}
 \int_{Y_S}\partial_t u_0(x)\operatorname{d}y = \sum_{i=1}^n\sum_{j=1}^n \partial_{x_i}\partial_{x_j}u_0(x) \hat D_{ij}+D_V \int_\Gamma (\nabla_x u_1 + \nabla_y u_2) \cdot \nu \operatorname{d} \Gamma(y)
 \end{align}

 We consider now the boundary condition. Using the ansatz function we get:
 
 \begin{align*}
k(u_0 + \varepsilon u_1 + \varepsilon^2 u_2 - \bar{u}) = D_V(&\varepsilon^{-1} \nabla_y u_0 \\
+& \varepsilon^0(\nabla_x u_0 + \nabla_y u_1)\\
+& \varepsilon ( \nabla_x u_1 + \nabla_y u_2)\\
+& \varepsilon^2 \nabla_x u_2) \cdot \nu
\end{align*}

Assume $k = \mathcal{O}(\varepsilon)$:
\begin{align}
 \varepsilon^{-1}D_V(\nabla_y u_0(x)) \cdot \nu &= 0\\
  D_V(\nabla_x u_0 + \nabla_y u_1) \cdot \nu &= 0\\
   \varepsilon D_V(\nabla_x u_1 + \nabla_y u_2) \cdot \nu&= k(u_0- \bar{u}) \\
   \varepsilon^2 D_V(\nabla_x u_2) \cdot \nu &= \varepsilon k u_1 \\
   0 &= \varepsilon^2 k u_2
\end{align}

Applying the boundary condition to \eqref{integral} and with the assumption $k \approx \varepsilon$ we get:

 \begin{align}
 \label{integral2}
 \int_{Y_S}\partial_t u_0(x)\operatorname{d}y &= \sum_{i=1}^n\sum_{j=1}^n \partial_{x_i}\partial_{x_j}u_0(x) \hat D_{ij}+D_V \int_\Gamma (\nabla_x u_1 + \nabla_y u_2) \cdot \nu \operatorname{d} \Gamma(y)\\
  &= \sum_{i=1}^n\sum_{j=1}^n \partial_{x_i}\partial_{x_j}u_0(x) \hat D_{ij}+\int_\Gamma u_0(x) - \bar{u} \operatorname{d} \Gamma(y)\\
  &= \sum_{i=1}^n\sum_{j=1}^n \partial_{x_i}\partial_{x_j}u_0(x) \hat D_{ij}
  + \mid\Gamma\mid(u_0 - \bar{u})
 \end{align}




Assume $k$ is independent of $\varepsilon$:
 \begin{align}
 \varepsilon^{-1}D_V(\nabla_y u_0(x)) \cdot \nu &= 0\\
 D_V(\nabla_x u_0 + \nabla_y u_1) \cdot \nu &= k(u_0- \bar{u}) \\
 \varepsilon D_V(\nabla_x u_1 + \nabla_y u_2) \cdot \nu &= \varepsilon k u_1 \\
 \varepsilon^2 D_V(\nabla_x u_2) \cdot \nu &= \varepsilon^2 k u_2
 \end{align}
 
 With this assumption we get:
 \begin{align}
 \label{integral2}
 \int_{Y_S}\partial_t u_0(x)\operatorname{d}y &= \sum_{i=1}^n\sum_{j=1}^n \partial_{x_i}\partial_{x_j}u_0(x) \hat D_{ij}+D_V \int_\Gamma (\nabla_x u_1 + \nabla_y u_2) \cdot \nu \operatorname{d} \Gamma(y)\\
  &= \sum_{i=1}^n\sum_{j=1}^n \partial_{x_i}\partial_{x_j}u_0(x) \hat D_{ij}+\int_\Gamma ku_1(x,y) \operatorname{d} \Gamma(y)\\
 \end{align}
 
 
%  Applying the boundary condition we get:
%  \begin{align*}
%   D_V \int_{Y_S} \nabla_y \cdot (\nabla_x u_1 + \nabla_y u_2) \operatorname{d} y = k \int_\Gamma u_1 \cdot \nu \operatorname{d} \Gamma(y)
%  \end{align*}
%  
%  The first two integrals together with the representation of $u_1$ as linear combination in \eqref{lin_combi_u_1} become:
%  
%  \begin{align*}
%  &D_V(\int_{Y_S} \nabla_x\cdot \nabla_x u_0 \operatorname{d}y + \int_{Y_S} \nabla_x \cdot \nabla_y u_1 \operatorname{d}y)\\
%  =& D_V(\nabla_x \cdot \nabla_x u_0 \int_{Y_S} \operatorname{d}y + \int_{Y_S} \nabla_x \cdot \nabla_y \sum_{j=1}^{n} w_j(y)\partial_{x_j} u_0(x) \operatorname{d} y)
%  \end{align*}
 
%  Since we consider $D_V=const$ we cannot describe $u_1$ in terms of $u_0$.
 
%  \textbf{Is $k = \mathcal{O}(\varepsilon$)?} 

% 
% 
% \begin{align*}
% k(u_0 + \varepsilon u_1 + \varepsilon^2 u_2 - \bar{u}) = D_V(&\varepsilon^{-1} \nabla_y u_0 \\
% +& \varepsilon^0(\nabla_x u_0 + \nabla_y u_1)\\
% +& \varepsilon ( \nabla_x u_1 + \nabla_y u_2)\\
% +& \varepsilon^2 \nabla_x u_2) \cdot \nu
% \end{align*}
 
%  Assume $k$ is independent of $\varepsilon$.
 
%  \begin{align}
%  0 = D_V \varepsilon^{-1} \nabla_y u_0(x,y) \cdot \nu \\
%  \Rightarrow u_0(x,y) = u_0(x) \text{ or } \nabla_y u_0(x,y) \perp \nu
%  \end{align}
%  
 
 
 
 
%  Assume now $k=c\varepsilon$:
 
 \end{document}
